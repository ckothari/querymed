% This is "www2010-sample.tex" copied from "www2005-sample.tex" V1.2 January 26 2004
% This file should be compiled with V1.4 of "www2010-submission.class"
%
% This example file demonstrates the use of the 'www2010-submission.cls'
% V1.4 LaTeX2e document class file. It is for those submitting
% articles to the WWW'04 Conference WHO DO NOT WISH TO 
% STRICTLY ADHERE TO THE SIGS (PUBS-BOARD-ENDORSED) STYLE.
% The 'www2010-submission.cls' file will produce a similar-looking,
% albeit, 'tighter' paper resulting in, invariably, fewer pages.
%
% ----------------------------------------------------------------------------------------------------------------
% This .tex file (and associated .cls V1.4) produces:
%       1) NO Permission Statement
%       2) WWW'04-specific conference (location) information
%       3) The Copyright Line with ACM data
%       4) NO page numbers
%
% ---------------------------------------------------------------------------------------------------------------
% This .tex source is an example which *does* use
% the .bib file (from which the .bbl file % is produced).
% REMEMBER HOWEVER: After having produced the .bbl file,
% and prior to final submission, you *NEED* to 'insert'
% your .bbl file into your source .tex file so as to provide
% ONE 'self-contained' source file.
%
% ================= IF YOU HAVE QUESTIONS =======================
% Questions regarding the SIGS styles, SIGS policies and
% procedures, Conferences etc. should be sent to
% Julie Goetz (goetz@acm.org) or Adrienne Griscti (griscti@acm.org)
%
% Technical questions only to
% Gerald Murray (murray@acm.org)
% ===============================================================
%
% For tracking purposes - this is V1.2 - January 26 2004
\documentclass{www2010-submission}

\begin{document}
%
\title{QueryMed: An Intuitive Federated SPARQL Query Builder for Biomedical RDF Data}
%\subtitle{[Extended Abstract]
%\titlenote{A full version of this paper is available as
%\textit{Author's Guide to Preparing ACM SIG Proceedings Using
%\LaTeX$2_\epsilon$\ and BibTeX} at
%\texttt{www.acm.org/eaddress.htm}}}
%
% You need the command \numberofauthors to handle the "boxing"
% and alignment of the authors under the title, and to add
% a section for authors number 4 through n.
%
% Up to the first three authors are aligned under the title;
% use the \alignauthor commands below to handle those names
% and affiliations. Add names, affiliations, addresses for
% additional authors as the argument to \additionalauthors;
% these will be set for you without further effort on your
% part as the last section in the body of your article BEFORE
% References or any Appendices.

\numberofauthors{2}
%
% Put no more than the first THREE authors in the \author command

% NOTE: All authors should be on the first page. For instructions
% for more than 3 authors, see:
% http://www.acm.org/sigs/pubs/proceed/sigfaq.htm#a18

\author{
%
% The command \alignauthor (no curly braces needed) should
% precede each author name, affiliation/snail-mail address and
% e-mail address. Additionally, tag each line of
% affiliation/address with \affaddr, and tag the
%% e-mail address with \email.
\alignauthor Oshani Seneviratne\\
       \affaddr{Massachusetts Institute of Technology}\\
       \affaddr{Cambridge, MA}\\
       \affaddr{USA}\\
       \email{oshani@csail.mit.edu}
\alignauthor Rachel Sealfon\\
       \affaddr{Massachusetts Institute of Technology}\\
       \affaddr{Cambridge, MA}\\
       \affaddr{USA}\\
       \email{rsealfon@csail.mit.edu}
}
\date{15 Feb 2010}

\maketitle
\begin{abstract}
We have developed an open-source SPARQL query builder and result set visualizer for biomedical data, QueryMed, that allows end users to easily construct and run translational medicine queries across multiple data sources. QueryMed is flexible enough to allow queries relevant to a wide range of biomedical topics, runs federated queries across multiple SPARQL endpoints, and is designed to be accessible to users who do not know the structure of the underlying ontologies or the SPARQL query language. The system allows users to select the data sources that they wish to use, drawing on their specialized domain knowledge to decide the most appropriate data sources to query.  Users can add additional data sources if they are interested in querying endpoints that are not in the default list. After retrieval of the initial result set, query results can be filtered to improve their relevance. As an advanced querying feature, the system also allows the user to exploit the underlying structure of the RDF data to improve query results. 
\end{abstract}

% A category with only the three required fields
\category{J.3}{Life and Medical Sciences}{Computer Applications}
\category{H.3.3}{Information Search and Retrieval}{Information Systems}

\keywords{Biomedical Ontologies, SPARQL, Query Federation, Query Building, Semantic Web, User Interfaces}

\section{Introduction}

RDF data in the semantic web, like the data in relational databases, is modeled in a structured format that can be queried using a standardized query language. The biomedical domain is among the early successes of the semantic web, due to the rapidity with which the biomedical community has made its data available in RDF triple stores. However, although a plethora of useful biomedical data is currently available in RDF, there is a need for easy-to use systems that do not require the end user to have knowledge of the underlying  structure of the data, and that also allow users to run federated queries on multiple SPARQL endpoints.  

For example, a physician may know her patient�s personal information, symptoms, current medications, and genotype. She may wish to determine the patient�s treatment plan and identify clinical trials for which the patient is eligible.  Although the physician has a single question--based on the information I have about this patient, what is the best treatment plan and set of clinical trials available?--there is no single data source that the physician can use to answer this question. The information that the physician needs must be gathered from numerous data sources such as Pubmed, DailyMed, Drugbank, LinkedCT, Diseasome,  and GO [Pubmed, DailyMed, Drugbank, LinkedCT, Diseasome, Gene Ontology]. Her question must be broken up into discrete pieces that can be executed individually at one data source at a time.  For example, to address her question, the physician might be interested in information on �coronary artery disease� (which she thinks is the cause of her patient�s symptoms). The information about the disease can be found in Diseasome. She might also want to know the set of drugs in DailyMed and Drugbank that can be used to treat coronary artery disease and that also will not react adversely with the patient�s current medications.  Information on clinical trials available for her patient can be found in LinkedCT.  Due to the large number of databases  that the physician needs to query in order to find an answer to her single question, she is likely to find useful a system that can automatically run queries over multiple data sources. Also, the physician may not know SPARQL query syntax, the location of the SPARQL endpoints, or the structure of the relevant ontologies. She is likely to want an intuitive way to query and to display the query result.  Developing intuitive ways to query multiple data sources and display results is both an important and a challenging problem.

This paper is organized as follows: Section 2 provides background information on the semantic web and its relevance for the biomedical domain. Section 3 describes our system. Section 4 discusses related work and illustrates how QueryMed differs from previous systems. Finally, section 5 outlines future work and summarizes the contributions of our system.


\section{Background}

\subsection{Biomedical Ontologies and Semantic Web}

The semantic web can be viewed as global database system for the data available on the world wide web. Semantic web data is modeled by structured languages such as RDF and OWL, and can be queried using a standardized query language, SPARQL [Berners-Lee, 2001]. The addition of structure to web data is intended to allow inferences to be automatically made by intelligent agents drawing on data from multiple sources. 

RDF data is described by a directed, labeled graph, with relationships among data items represented  in triples containing a subject, an object and a predicate.  For example, we might want to express the concept, "Sam is a professor for 6.830." The corresponding triple expresses the relationship "is a professor for" between two entities, �Sam� and �6.830�. The relationship "is a professor for" is known as the predicate of the triple; "Sam" is the subject while "6.830" is the object.  In order for this statement to be machine readable, each of the entities in the statement ("Sam," "is a professor for," and "6.830") will be described by a Uniform Resource Identifier (URI) [Manola, 2004].  The Web Ontology Language (OWL) is a language that permits the definition of ontologies, allowing additional properties and constraints on relationships among data to be represented [Smith, 2004]. Some parallels between relational databases and RDF data on the semantic web are illustrated in table 1 [Berners-Lee, 1998].

Although most web data are not currently available in semantic web formats, the biomedical community has been extremely proactive in making structured biomedical data available in RDF. Thus, the biomedical knowledge domain has been an early success of the semantic web [Yip, 2009]. Many major biological and biomedical data resources, including Gene Ontology, DailyMed, LinkedCT, and Diseasome, are currently available as RDF triplestores.   Integrating biomedical data across multiple data sources and automatically extracting specific  knowledge from web resources are crucial tasks for physicians and biologists, and these semantic web resources represent valuable standardized repositories of information that can be automatically mined for biological knowledge.  The Semantic Web Health Care and Life Sciences Interest Group (HCLSIG) has been formed with the purpose of exploring the applications of the semantic web to the biomedical domain [HCLSIG].

However, although many valuable resources in the biomedical domain are available in RDF, there are a number of challenges that must be addressed in order to make such resources of maximal use to physicians, patients, and life scientists.  One challenge is constructing systems that allow end users to run intuitive queries on biomedical data.  Users of biomedical resources are likely to have extensive domain knowledge, but be unfamiliar with the SPARQL query language syntax and with the structure of biomedical ontologies. It is important to design user-friendly systems that allow these users to take advantage of the wealth of structured biological knowledge available on the semantic web.  Another central challenge is designing systems that permit users to query multiple data sources simultaneously, since relevant biological data is often distributed among many sources [Pasquier, 2008].

\subsection{SPARQL Query Language}

A SPARQL query performs a pattern-matching search on the underlying RDF graph of the desired resource.  Most SPARQL queries are constructed from basic graph patterns.  Like RDF triples, a basic graph pattern contains a subject, an object, and a predicate; however, in the basic graph pattern, the subject, object, or predicate may be replaced by variables.  The query result is the set of tuples matching the query pattern [Prud'hommeaux, 2008].

The SPARQL query language is designed to query semantic web data, just as SQL is designed to query relational data. However unlike SQL, SPARQL supports multiple query forms appropriate for  retrieving information from structured graph data. In SPARQL, the following types of queries are possible:
1. "ASK" query to test whether a graph contains  specific data of interest.  This type of query returns a boolean result.
2. "DESCRIBE" query to generate an RDF description of one or more resources.  This type of query returns a set of RDF triples.
3. "CONSTRUCT" query to create a custom RDF graph based on a query criteria. This can also be used to transform RDF data.  This type of query returns a graph matching the query criteria.
4. �SELECT� query to identify triples matching a query pattern.  This type of query returns a set of matching triples.

Our system supports a subset of queries of the �SELECT� type.

\subsection{The Query Federation Problem}

In the relational database world, data is contained within "data silos."  Schemas for different databases are likely to differ, even if the data that is stored is similar.  For example, suppose that two companies, Company A and Company B, construct independent databases,   Database 1 and Database 2 respectively, to describe all known drugs.  Company A may include �brand name,� �generic name,� �active compound,� �cost,� �indications,� and �protein target� information in its schema, while Company B may define the corresponding table in its database using a different schema.  In addition to discrepancies in table schemas, the data itself may be represented differently by the two companies. For example, the protein target of a drug to treat cardiovascular disease may be described as �ABCA1� in Database 1, and by a synonym, �ABC1,� in Database 2.  Therefore, it will be difficult to integrate the data between Database 1 and Database 2 without writing extensive application logic, even though the two databases contain highly overlapping information. In contrast, RDF data is represented in a form that is amenable for data integration spanning multiple data sources, because unique entities are described by unique Uniform Resource Identifiers (URIs). 

However, because of the scale free nature of the web, locating the appropriate data endpoint to execute a specific query is a challenging problem. This is known as the query federation problem. While the SPARQL query language has some support for expressing queries that span multiple data sources, there is a need for improved systems able to orchestrate queries across multiple sources. 

\section{QueryMed}

\subsection{Architecture overview}

A general overview of QueryMed architecture is shown in Figure 1. The main components of the system are the user interface and the proxy server that retrieves biomedical data from remote SPARQL endpoints. 

QueryMed is implemented in Java in the backend and JavaScript in the frontend.  The JQuery library [Resig] was used to develop an attractive user interface.  In the backend, the Jena library [McBride] is used to run the SPARQL queries. QueryMed relies on the following remote resources:
SPARQL Endpoints:  QueryMed runs queries on multiple remote SPARQL endpoints that expose biomedical data.
Sources List: The default list of endpoints available to the user is stored in a source list on the proxy server.

\begin{figure}
\centering
%\epsfig{file=fly.eps, height=1in, width=1in}
\caption{QueryMed Architecture Overview}
\end{figure}


\subsection{}



\section{}

\subsection{}


\subsection{}



\subsection{Tables}

\begin{table}
\centering
\caption{Frequency of Special Characters}
\begin{tabular}{|c|c|l|} \hline
Non-English or Math&Frequency&Comments\\ \hline
\O & 1 in 1,000& For Swedish names\\ \hline
$\pi$ & 1 in 5& Common in math\\ \hline
\$ & 4 in 5 & Used in business\\ \hline
$\Psi^2_1$ & 1 in 40,000& Unexplained usage\\
\hline\end{tabular}
\end{table}

To set a wider table, which takes up the whole width of
the page's live area, use the environment
\textbf{table*} to enclose the table's contents and
the table caption.  As with a single-column table, this wide
table will ``float" to a location deemed more desirable.
Immediately following this sentence is the point at which
Table 2 is included in the input file; again, it is
instructive to compare the placement of the
table here with the table in the printed dvi
output of this document.


\begin{table*}
\centering
\caption{Some Typical Commands}
\begin{tabular}{|c|c|l|} \hline
Command&A Number&Comments\\ \hline
\texttt{{\char'134}alignauthor} & 100& Author alignment\\ \hline
\texttt{{\char'134}numberofauthors}& 200& Author enumeration\\ \hline
\texttt{{\char'134}table}& 300 & For tables\\ \hline
\texttt{{\char'134}table*}& 400& For wider tables\\ \hline\end{tabular}
\end{table*}

\subsection{Figures}
Like tables, figures cannot be split across pages; the
best placement for them
is typically the top or the bottom of the page nearest
their initial cite.  To ensure this proper ``floating'' placement
of figures, use the environment
\textbf{figure} to enclose the figure and its caption.

This sample document contains examples of \textbf{.eps}
and \textbf{.ps} files to be displayable with \LaTeX.  More
details on each of these is found in the \textit{Author's Guide}.

\begin{figure}
\centering
%\epsfig{file=fly.eps}
\caption{A sample black and white graphic (.eps format).}
\end{figure}

\begin{figure}
\centering
%\epsfig{file=fly.eps, height=1in, width=1in}
\caption{A sample black and white graphic (.eps format)
that has been resized with the \texttt{epsfig} command.}
\end{figure}


As was the case with tables, you may want a figure
that spans two columns.  To do this, and still to
ensure proper ``floating'' placement of tables, use the environment
\textbf{figure*} to enclose the figure and its caption.
\begin{figure*}
\centering
%\epsfig{file=flies.eps}
\caption{A sample black and white graphic (.eps format)
that needs to span two columns of text.}
\end{figure*}
and don't forget to end the environment with
{figure*}, not {figure}!

Note that either {\textbf{.ps}} or {\textbf{.eps}} formats are
used; use
the \texttt{{\char'134}epsfig} or \texttt{{\char'134}psfig}
commands as appropriate for the different file types.

\begin{figure}
\centering
%\psfig{file=rosette.ps, height=1in, width=1in,}
\caption{A sample black and white graphic (.ps format) that has
been resized with the \texttt{psfig} command.}
\end{figure}

\subsection{Theorem-like Constructs}
Other common constructs that may occur in your article are
the forms for logical constructs like theorems, axioms,
corollaries and proofs.  There are
two forms, one produced by the
command \texttt{{\char'134}newtheorem} and the
other by the command \texttt{{\char'134}newdef}; perhaps
the clearest and easiest way to distinguish them is
to compare the two in the output of this sample document:

This uses the \textbf{theorem} environment, created by
the \texttt{{\char'134}newtheorem} command:
\newtheorem{theorem}{Theorem}
\begin{theorem}
Let $f$ be continuous on $[a,b]$.  If $G$ is
an antiderivative for $f$ on $[a,b]$, then
\begin{displaymath}\int^b_af(t)dt = G(b) - G(a).\end{displaymath}
\end{theorem}

The other uses the \textbf{definition} environment, created
by the \texttt{{\char'134}newdef} command:
\newdef{definition}{Definition}
\begin{definition}
If $z$ is irrational, then by $e^z$ we mean the
unique number which has
logarithm $z$: \begin{displaymath}{\log e^z = z}\end{displaymath}
\end{definition}

Two lists of constructs that use one of these
forms is given in the
\textit{Author's  Guidelines}.

There is one other similar construct environment, which is
already set up
for you; i.e. you must \textit{not} use
a \texttt{{\char'134}newdef} command to
create it: the \textbf{proof} environment.  Here
is a example of its use:
\begin{proof}
Suppose on the contrary there exists a real number $L$ such that
\begin{displaymath}
\lim_{x\rightarrow\infty} \frac{f(x)}{g(x)} = L.
\end{displaymath}
Then
\begin{displaymath}
l=\lim_{x\rightarrow c} f(x)
= \lim_{x\rightarrow c}
\left[ g{x} \cdot \frac{f(x)}{g(x)} \right ]
= \lim_{x\rightarrow c} g(x) \cdot \lim_{x\rightarrow c}
\frac{f(x)}{g(x)} = 0\cdot L = 0,
\end{displaymath}
which contradicts our assumption that $l\neq 0$.
\end{proof}

Complete rules about using these environments and using the
two different creation commands are in the
\textit{Author's Guide}; please consult it for more
detailed instructions.  If you need to use another construct,
not listed therein, which you want to have the same
formatting as the Theorem
or the Definition\cite{salas:calculus} shown above,
use the \texttt{{\char'134}newtheorem} or the
\texttt{{\char'134}newdef} command,
respectively, to create it.

\subsection*{A {\secit Caveat} for the \TeX\ Expert}
Because you have just been given permission to
use the \texttt{{\char'134}newdef} command to create a
new form, you might think you can
use \TeX's \texttt{{\char'134}def} to create a
new command: \textit{Please refrain from doing this!}
Remember that your \LaTeX\ source code is primarily intended
to create camera-ready copy, but may be converted
to other forms -- e.g. HTML. If you inadvertently omit
some or all of the \texttt{{\char'134}def}s recompilation will
be, to say the least, problematic.

\section{Conclusions}


\section{Acknowledgments}

This section is optional; it is a location for you
to acknowledge grants, funding, editing assistance and
what have you.  In the present case, for example, the
authors would like to thank Gerald Murray of ACM for
his help in codifying this \textit{Author's Guide}
and the \textbf{.cls} and \textbf{.tex} files that it describes.

%
% The following two commands are all you need in the
% initial runs of your .tex file to
% produce the bibliography for the citations in your paper.
\bibliographystyle{abbrv}
\bibliography{sigproc}  % sigproc.bib is the name of the Bibliography in this case
% You must have a proper ".bib" file
%  and remember to run:
% latex bibtex latex latex
% to resolve all references
%
% ACM needs 'a single self-contained file'!
%
%APPENDICES are optional
%\balancecolumns
\appendix
%Appendix A
\section{Headings in Appendices}
The rules about hierarchical headings discussed above for
the body of the article are different in the appendices.
In the \textbf{appendix} environment, the command
\textbf{section} is used to
indicate the start of each Appendix, with alphabetic order
designation (i.e. the first is A, the second B, etc.) and
a title (if you include one).  So, if you need
hierarchical structure
\textit{within} an Appendix, start with \textbf{subsection} as the
highest level. Here is an outline of the body of this
document in Appendix-appropriate form:
\subsection{Introduction}
\subsection{The Body of the Paper}
\subsubsection{Type Changes and  Special Characters}
\subsubsection{Math Equations}
\paragraph{Inline (In-text) Equations}
\paragraph{Display Equations}
\subsubsection{Citations}
\subsubsection{Tables}
\subsubsection{Figures}
\subsubsection{Theorem-like Constructs}
\subsubsection*{A Caveat for the \TeX\ Expert}
\subsection{Conclusions}
\subsection{Acknowledgments}
\subsection{Additional Authors}
This section is inserted by \LaTeX; you do not insert it.
You just add the names and information in the
\texttt{{\char'134}additionalauthors} command at the start
of the document.
\subsection{References}
Generated by bibtex from your ~.bib file.  Run latex,
then bibtex, then latex twice (to resolve references)
to create the ~.bbl file.  Insert that ~.bbl file into
the .tex source file and comment out
the command \texttt{{\char'134}thebibliography}.
%
% This next section command marks the start of
% Appendix B, and does not continue the present hierarchy
\section{More Help for the Hardy}
The www2010-submission.cls file itself is chock-full of succinct
and helpful comments.  If you consider yourself a moderately
experienced to expert user of \LaTeX, you may find reading
it useful but please remember not to change it.
\balancecolumns % GM July 2000
% That's all folks!
\end{document}
